\section{Examples}
\label{s:Examples}

This appendix provides some additional hints and examples for the
layout and style of the thesis. It is worthwhile to look at the source
file \verb|Examples.tex| for this appendix to understand how it was
created.



\subsection{Tables}

Tables are left justified and the caption appears on top as seen in
Table~\ref{t:Translations}.

\begin{table}[ht] % ht = here & top
\caption[Translations]{\label{t:Translations}Translations.}
\centering
\begin{tabular}{ll}
\hline
\textbf{English} & \textbf{German}\\
\hline
cell phone       & Handy\\
Diet Coke        & Coca Cola light\\
\hline
\end{tabular}
\end{table}



\subsection{Figures}

Figure~\ref{f:IRISlogo} shows a simple figure with a single picture
and Figure~\ref{f:SubfigureExample} shows a more complex figure
containing subfigures.

\begin{figure}[ht]
\centering
\includegraphics[width=.6\linewidth]{files/UZHlogo}
\caption[IRIS logo]{\label{f:IRISlogo}IRIS logo.}
\end{figure}

\begin{figure}[ht]
\centering
\subfigure[ETH logo]{\includegraphics[height=12mm]{files/ETHlogo}}\quad
\subfigure[IRIS logo]{\includegraphics[height=12mm]{files/UZHlogo}}
\caption[Subfigure example]{\label{f:SubfigureExample}Two pictures as
  part of a single figure through the magic of the subfigure package.}
\end{figure}



\subsection{Units}

The SIUnits package provides nice spacing for units as demonstrated in
Table~\ref{t:SIUnits}. Use of the package also makes it easy to change
the style or even the unit text in the future.

\begin{table}[ht]
\caption[Spacing for units]{\label{t:SIUnits}Spacing for units.}
\centering
\begin{tabular}{ll}
\hline
\textbf{Output}   & \textbf{Command}\\
\hline
42m               & \verb|42m|\\
\unit{42}{\metre} & \verb|\unit{42}{\metre}|\\
42 m              & \verb|42 m|\\
\hline
\end{tabular}
\end{table}



\subsection{Miscellany}

\begin{description}

\item[Capitalization.] When referring to a named table (such as in the
  previous section), the word \emph{table} is capitalized. The same is
  true for figures, chapters and sections.

\item[Naming of structural elements.] Refer to a \verb|\section| in
  \LaTeX\ as a chapter and call a \verb|\subsection| section. (I don't
  like the way \verb|\chapter|s are rendered in the report document
  class. Hence the suboptimal markup/naming correspondence.)

\item[Bibliography.] Use \verb|bibtex| to make your life easier and to
  produce consistently formatted entries.

\item[Contractions.] Avoid contractions. For instance, use ``do not''
  rather than ``don't.''

\item[Captions.] A brief version of a caption can be provided for the
  list of figures and tables as demonstrated with the caption of
  Figure~\ref{f:SubfigureExample}. The mechanism can also be used to
  get rid of the final period of a caption in the lists.

\end{description}
